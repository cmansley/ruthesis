%%%%%%%%%%%%%%%%%%%%%%%%%%%%%%%%%%%%%%%%%%%%%%%%%%%%%%%%%%%%%%%%%%%%%%%%%%%%%%%%
%% manySubfigsheader.tex
%% 2009.01.09 - tedmunds
%%
%% This file is the header file used to perform chapter-specific configuration
%% for one chapter in the dissertation template.
%% This file is for preamble commands that are needed by the chapter (but not
%% generally needed by the document as a whole).
%% There is another file (./manySubfigsChapterHeader) that can be used to
%% perform non-preamble configuration at the start of the chapter.
%% Note that the commands issued (and packages used) here may not _conflict_
%% with the rest of the document; they should play nicely.
%% This file is included by the main header (../header/header.tex).
%%
%%%%%%%%%%%%%%%%%%%%%%%%%%%%%%%%%%%%%%%%%%%%%%%%%%%%%%%%%%%%%%%%%%%%%%%%%%%%%%%%

%% This chapter uses the graphicx package for figures.
\usepackage{graphicx}
%% This chapter uses the subfig package to break a figure into parts (on
%% separate pages) while maintaining the sub-figure labels.
\usepackage{subfig}
%% The alphalph package allows us to have more than 26 things in an alphabetically numbered
%% list
\usepackage{alphalph}

%% Replace the default \alph command with \alphalph (from the alphalph package) so that
%% \alph can handle values higher than 26.  This is useful in situations where you have
%% more than 26 subfigures in a figure.  Or more than 26 appendices.
\renewcommand{\alph}[1]{\alphalph{\value{#1}}} 