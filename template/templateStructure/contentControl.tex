As described in \autoref{sec:docClass}, the \texttt{ruthesis} document class includes a number of options that can be used to control the typesetting of the manuscript.
However, there are any number of aspects of the document that the author may wish to control throughout the development process.
For this reason, it is useful to define author-set switches in a central location to control the typesetting of the dissertation.
For example, if a committee member mentions that he doesn't like references at the end of each chapter, it is handy to be able to change a single switch from true to false, rather than having to go digging through multiple files to comment out all the source code that generates per-chapter bibliographies.

This template defines a number of such switches at the beginning of the root source file (in the preamble).
The purpose of each switch is described in the adjoining comments, and some of the switches are mentioned specifically at various places in this document.

One set of switch is used specifically to control the inclusion of content.
Dissertations tend to be long documents, and it is often beneficial to be able to typeset just a portion (a chapter or two).
To control the inclusion of content, this template defines a switch for each chapter (e.g., $\backslash$\texttt{ifTemplateStructure} for this chapter);
then whenever commands are issued that depend on the assumption that the chapter is to be typeset (like the $\backslash${\texttt{import}} command that includes the chapter), the commands are embedded in a check of the switch.
For example:
\begin{quotation}%
\begin{verbatim}
\ifTemplateStructure%
  \chapter{Dissertation Template}%
  \label{sec:templateStructure}%
  \import{templateStructure/}{templateStructure}%
\else\fi% end ifTemplateStructure
\end{verbatim}%
\end{quotation}%
Similar switches are used to control the inclusion of the C.V., the Acknowledgements, etc.
 