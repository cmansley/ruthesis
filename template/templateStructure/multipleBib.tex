In \autoref{sec:templateStructure:natbib} we described how \texttt{natbib} can be used to create attractive citations and bibliographies while satisfying some of the Graduate School's style dictums.
However, we did not address how to achieve the ``references at the end of each chapter'' that the style guide refers to as one option.
Even when only a single bibliography at the end of the dissertation is produced, this template uses multiple bibliographies, since the C.V.\ publication list is produced by \BibTeX\ (now would be a good time to mention that in addition to the two \texttt{.bst} files discussed in \autoref{sec:templateStructure:natbib:bst}, this bundle also includes a \href{../bibtexery/ruvita.bst}{\BibTeX\ style file for chronological publication lists}).

The default behaviour of \LaTeX\ is to handle a single bibliography in a document, but there are a number of \LaTeX\ packages that provide support for multiple bibliographies (e.g., \texttt{chapterbib}, \texttt{multibib}).
The package used by this template is \texttt{bibunits}; this package has been found to work well with \texttt{natbib}, \texttt{hyperref}, and the template's structure.
The \texttt{bibunits} package works by encapsulating portions of the document in a \texttt{bibunit} environment; within that \texttt{bibunit} environment, the $\backslash$\texttt{putbib} command can be used to produce a bibliography containing the references cited within the \texttt{bibunit}.
E.g.:
\begin{quotation}%
\begin{verbatim}
\begin{bibunit}%
%% Include the contents of a chapter
\import{}{someChapter}%
%% Create a bibliography section (defined by ruthesis.cls) at
%% the end of the chapter
\begin{bibliographysection}%
%% Specify each of the BibTeX database files that are
%% referenced in this chapter.
\putbib[bibliography/neuroControl,%
        bibliography/basketWeaving]%
\end{bibliographysection}%
\end{bibunit}%
\end{verbatim}%
\end{quotation}%

In this template, the \texttt{bibunits} package is always used (since the C.V.\ publication list is produced as one \texttt{bibunit}), but the inclusion of a References section at the end of each chapter is controlled by the \texttt{perchapterbib} boolean defined at the beginning of the \href{./dissertation.tex}{root source file}.

The \texttt{bibunits} package also supports the generation of an overall bibliography that contains copies of the references from each of the \texttt{bibunits}.
This feature is used to generate a bibliography at the end of the dissertation (if called for by the \texttt{endbib} boolean defined in the root source file).
Note that the references in the C.V.\ publication list are excluded from this overall bibliography by calling the \texttt{bibunits} $\backslash$\texttt{globalcitecopyfalse} command within the \href{./backMatter/vita.tex}{C.V.\ \texttt{bibunit}}.

\subsection{A Word About Editors}
Many IDE-like \LaTeX\ editors try to help automatically compile a document by executing a set pattern of commands (like \texttt{latex}, \texttt{bibtex}, \texttt{latex}, \texttt{latex}, \texttt{latex}), making certain filename assumptions (such as the assumption that \texttt{latex sourceName.tex} will produce a single auxiliary file, \texttt{sourceName.aux}, that contains all the document's citations, and that \texttt{bibtex} only needs to be executed on that one file).
Beware of such assumptions; the \texttt{bibunits} package produces a separate \texttt{.aux} file for each \texttt{bibunit} (called \texttt{bu1.aux}, \texttt{bu2.aux}, \texttt{bu3.aux}, etc.).
The good news is that up-to-date versions of the \texttt{bibtex} executable seem to handle \texttt{bibunits}'s output correctly.
However, if the editor is trying to do its own processing to tell you what entries are in your bibliography, the editor needs to understand about \texttt{bibunits} as well.

\subsubsection{Windows (WinEdt)}
If you have an up-to-date version of \href{http://www.winedt.com}{WinEdt}, then it knows what to do.
To let WinEdt populate the list of references in your \BibTeX\ databases, include them all in a comment (one \% only).
E.g.:
\begin{quotation}%
\begin{verbatim}
%input "bibliography/neuroControl.bib"
%input "bibliography/basketWeaving.bib"
%input "bibliography/myPublications.bib"
\end{verbatim}%
\end{quotation}%

\subsubsection{OS X (TeXShop)}
Recent versions of \href{http://www.uoregon.edu/~koch/texshop/}{TeXShop} include \texttt{latexmk}, a collection of perl scripts for executing \texttt{latex} (and auxiliary programs like \texttt{bibtex}) the requisite number of times to resolve cross references, etc.
When the Typeset button is pressed, the command in the selection-list next to the button is executed; if \texttt{latexmk} is selected, all the necessary programs should be run (similarly with \texttt{pdflatexmk} for producing \texttt{.pdf}s directly)\footnote{Despite debugging efforts, using \texttt{latexmk} with TeXShop does not work with the PNG format image files used in this template.
The conditional code that has been included to make \texttt{latex} happy with \texttt{.png} and \texttt{.jpg} files (see \autoref{sec:templateStructure:compiling}) works on the Windows \texttt{latex}$\rightarrow$\texttt{dvi2ps}$\rightarrow$\texttt{ps2pdf} pipeline, but \texttt{latexmk} produces improper \texttt{.dvi} files.
Using \texttt{pdflatexmk} works.}.

\subsection{Troubleshooting}
If the use of the \texttt{bibunits} package causes problems, and per-chapter bibliographies are not required, the use of the package can be eliminated, as long as the C.V.\ publication list is typeset manually.
One way to do this is to create a separate \texttt{.tex} file with just one bibliography (the publication list), and copy the resulting \texttt{.bbl} file into the source for the C.V.
Of course, that entire process has to be re-executed any time the content or formatting of the publication list needs to be changed.