The Graduate School has \href{http://gsnb.rutgers.edu/guide.php3}{this} to say about the format of References, Citations, and Bibliography:
\begin{quote}
Footnotes at the bottom page[sic], endnotes at the ends of chapters or at the end of manuscript.
Number notes consecutively.
When notes are at the end of chapters, each chapter's notes should begin with the number one (1).
Be consistent throughout and conform to generally accepted practice in the discipline.
\end{quote}
There is a lot of room for self-contradiction in these instructions; for example, the generally accepted practice in your discipline may be to order references alphabetically, rather than by citation order.
Also, some disciplines may use author-date style citations, rather than numbers.
For this reason, this template distribution includes some different options for how to include bibliographies in the manuscript.

Of course, part of the appeal of preparing a manuscript in \LaTeX\ is that references can be automatically selected for inclusion from a database and typeset according to a \BibTeX\ style.
However, basic \BibTeX\ technology has been improved on by more powerful \LaTeX\ packages.
The specific package recommended (and supported by this template) is \texttt{natbib}.
The \texttt{natbib} package is a reimplementation of \LaTeX's $\backslash$\texttt{cite} functionality that allows for more configurable citation styles (such as author-year, numerical, annotated).
\texttt{natbib} also works with the \texttt{hyperref} package to create hyperlinks from citation to bibliography (and, with the \texttt{backref} option, from the bibliography to the citation).
From an author's perspective, the main change that comes with using the \texttt{natbib} package is that use of the $\backslash$\texttt{cite} command is replaced by either $\backslash$\texttt{citep} (for parenthetical citations --- like this~\citep{Edmunds:2005:ICDE}) or $\backslash$\texttt{citet} (for textual citations --- like those used by \citet{Kaufman:2005:SIGGRAPH}).
For more information on the citation capabilities of the \texttt{natbib} package, check out the \href{http://merkel.zoneo.net/Latex/natbib.php}{Reference sheet for \texttt{natbib} usage}.

\subsection{\BibTeX\ Style Files}
\label{sec:templateStructure:natbib:bst}%
As mentioned above, the Graduate School Style Guide does not specify one set style for the typesetting of citations and bibliographies (except that bibliographies must be single-spaced).
In some disciplines, numerical citations may be the ``generally accepted practice,'' while in others, the author-date style might be the norm.
Also, the information included in the bibliography entries might vary.
The typesetting of the bibliographic entries is controlled by a \BibTeX\ style file (\texttt{.bst}).
In order to be able to do things like author-year style citations, \texttt{natbib} requires special \texttt{.bst} files.
Included in this bundle are two different reference styles: \href{../bibtexery/ruthesis.bst}{\texttt{ruthesis.bst}} and \href{../bibtexery/ruthesisciteorder.bst}{\texttt{ruthesisciteorder.bst}}.

\texttt{ruthesis.bst} is a style file for use with author-year style citations; it produces a list of bibliography entries in alphabetical order (by first author's last name).
Since space constraints are not usually a problem in dissertation manu\-scripts, relatively little abbreviation is used in the typesetting of the bibliographic entries.

The second \BibTeX\ style file, \texttt{ruthesisciteorder.bst}, is a \texttt{natbib} friendly style that sorts the bibliography entries by the order of citation (to comply with the Graduate School's instruction that notes should be numbered consecutively).

As well as specifying the appropriate \BibTeX\ style, \texttt{natbib} has to be properly configured to select the desired citation/bibliography scheme.
In this template, the choice between using numerical citations (and hence sorting by citation order) or author-year citations is controlled by the \texttt{numericalcitation} boolean defined in the root source file.
By setting the value of this switch, the author can select from one of these two citation schemes.
(See \href{./header/header.tex}{\texttt{header/head\-er.tex}} to find out how the switch's value affects the setup of \texttt{natbib}.)

\subsubsection{Custom Styles}
Not all dissertation authors may be satisfied with these two bibliography style choices.
A particularly likely scenario is that an author may want to change how the bibliographic entries are typeset (for example, to list authors as Last Name, First Name).
Unfortunately, \BibTeX\ style files are not very easy to hand-edit.
The best solution for authors who need to change the bibliography style is probably to use the \texttt{custom-bib} package to generate a new bibliography (this is the method that was used to create the style files discussed above).

The \texttt{custom-bib} package provides an interactive process for creating a \texttt{.bst} file for the desired typesetting style.
The result of the interactive process is a batch job (\texttt{.dbj} file) that can be processed by \LaTeX\ to produce the \texttt{.bst} file.
For completeness, the batch job files that were used to produce \texttt{ruthesis.bst} and \texttt{ruthesisciteorder.bst} are also included in this distribution (the files are \href{../bibtexery/ruthesis.dbj}{\texttt{ruthesis.dbj}} and \href{../bibtexery/ruthesisciteorder.dbj}{\texttt{ruthesisciteorder.dbj}}, respectively).

Note that after the automated generation of the \texttt{.bst} files, some hand-editing was done (as noted in the comments at the beginning of the \texttt{.bst} files) to provide \texttt{hyperref}-compatible D.O.I.\ links.