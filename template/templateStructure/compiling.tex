There are a number of different applications (even on the same platform) for compiling \LaTeX\ source files into documents.
When the end goal is an Adobe Portable Document Format (\texttt{.pdf}) file, the two common choices are:
\begin{itemize}%
\item \texttt{pdflatex}: compile directly from the source to a \texttt{.pdf} file
\item \texttt{latex}, \texttt{dvitops}, \texttt{pstopdf}: \texttt{latex} compiles the source into a \texttt{.dvi} file, \texttt{dvi\-tops} converts the \texttt{.dvi} file to a \texttt{.ps} file, and \texttt{pstopdf} converts the \texttt{.ps} file to a \texttt{.pdf} file.
\end{itemize}%
The output of these two different options is often similar, but not exactly the same.
Also, depending on the method used, different source code is acceptable.
For instance, when using \texttt{pdflatex}, supported graphics formats are \texttt{.pdf}, \texttt{.png}, and \texttt{.jpg}, but when using \texttt{latex}, \texttt{.pdf} graphics files are not allowed (but \texttt{.eps} files are) and bounding box information is not automatically extracted from raster image file formats (like \texttt{.png} and \texttt{.jpg}).
Another difference is in support of line-wrapping links produced by the \texttt{hyperref} package.
\href{http://www.ctan.org/tex-archive/macros/latex/contrib/hyperref/README.pdf} {When compiled with the \texttt{pdftex} driver, \texttt{hyperref} links that would otherwise stick out into the margin (like this one) can be typeset on two lines (providing the link text can be broken, either naturally or according to $\backslash$\texttt{-} hints)}; however, as noted in the \texttt{hyperref} README, the \texttt{dvips} driver does not support that feature.
The \texttt{hyperref} package's \texttt{breaklinks} option stops the text from being typeset in the margin, but the active link area is misplaced when the electronic document is viewed in a PDF viewer.

The usual course of action is to pick one of these two compiling schemes (\texttt{pdflatex} is recommended), stick with it, and not worry about code that doesn't compile with the other method.
This option is particularly attractive for a dissertation, where the author does not have to worry about interoperability with a co-author's development environment.
To maintain generality though, this template has been designed to work with either of the two compiler schemes mentioned above.
To make this possible, some use has been made of the \texttt{ifpdf} package (which provides the $\backslash$\texttt{ifpdf} test that signals whether the document is being compiled directly to PDF).
Specifically, to deal with the graphics file format incompatibility mentioned above, a different graphics file extension list is used depending on the result of the $\backslash$\texttt{ifpdf} test.
When compiling directly to PDF, the command $\backslash$\texttt{DeclareGraphicsExtensions\{.png,.jpg,.pdf\}} is used to signal that $\backslash$\texttt{includegraphics\{someImage\}} commands should search for files named \texttt{someImage.png}, \texttt{someImage.jpg}, or \texttt{someImage.pdf}.
When compiling to an intermediate file format, $\backslash$\texttt{DeclareGraphicsExtensions\{.eps,.png,.jpg\}} is used.
To allow different chapters to have different variations of these search orders, the commands are issued at the beginning of chapters that include graphics (e.g., in \href{./usefulTechniques/chapterHeader.tex}{usefulTechniques/chapterHeader.tex} and \href{./appendices/manySubfigsChapterHeader.tex}{appendices/many\-SubfigsChapterHeader.tex}). 