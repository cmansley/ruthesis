Here is an example of the use of the commands defined in the headers to make consistent typesetting easy.
We defined a macro to make it easy to typeset this quantity, \proxy, whether in math-mode or not.
We also defined its derivative, \proxyVel.

The \texttt{amsmath} package provides the \texttt{align} environment that is nice for typesetting multi-row equations:
\begin{align}
\proxy &= \master \\
\userForce &= 0
\end{align}

The \texttt{align} environment can be used for more complicated equations too:
\begin{align}
\userForce &= \left\{\begin{array}{cl}
  k(\floor - \master), & \master < \floor\\
  0, & \master \geq \floor
  \end{array}\right. \\
\proxy &= \left\{\begin{array}{cl}
  \floor, & \master < \floor\\
  \master, & \master \geq \floor
  \end{array}\right.
\end{align}

