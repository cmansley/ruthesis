\begin{floatingtable}[r]{%
\captionsetup[subfloat]{labelformat=empty}%
  \centering%
    \begin{tabular}{|r|l|}%
      \hline%
      \multicolumn{2}{|c|}{Target}\\%
      \hhline{|*2{=}|}%
      $\mu_0$ & 0.1\\%
      \hline%
      $a_1$ & 0.783\\%
      \hline%
      $a_2$ & 0.1165\\%
      \hline%
      $\sigma$ & 0.05\\%
      \hline%
    \end{tabular}%
  \ %
    \begin{tabular}{|r|l|}%
      \hline%
      \multicolumn{2}{|c|}{Distractor}\\%
      \hhline{|*2{=}|}%
      $\mu_0$ & 0.1\\%
      \hline%
      $a_1$ & 0.25\\%
      \hline%
      $a_2$ & 0.1\\%
      \hline%
      $\sigma$ & 0.1\\%
      \hline%
    \end{tabular}%
  %% Remove a little vertical space to make things tidy
  \vspace{-0.5\baselineskip}%
  }% end of the argument to floatingtable
  %% Unfortunately, while the ruthesis document class fixes floats to be
  %% single-spaced, this environment is a floatingtable, not a float.
  %% Until someone adds functionality to the ruthesis document class, we
  %% will manually make the caption single-spaced:
  \begin{singlespace}%
  \caption[Texture Parameters]{%
    Autoregression parameters used to generate the target and distractor textures.
  }%
  \label{tab:textureParams}%
  \end{singlespace}%
\end{floatingtable}%
Given the relatively large column-width of this single-column dissertation format, some figures (or tables) are simply too small to be placed in the entire margin-to-margin space that is the default.
In this situation, one alternative is to find (or make) another small figure that fits well next to your problem child.
If that solution is not tenable, then another alternative is to place the figure within the text in such a way that the surrounding text wraps fluidly around the figure.
\autoref{tab:textureParams} is an example of this type of wrapped figure (or table).
Wrapped figures and wrapped tables can be created using the \texttt{floatflt} package. 