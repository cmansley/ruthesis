In this chapter, we give examples of some \LaTeX\ techniques that may be useful for producing pleasant dissertations.
In order to help this template better display the capabilities of the ruthesis document class (and recommended packages), some meaningless material has been arbitrarily left in.
This citation, for instance~\citep{Johansson:1996:HANDBRAIN}.

\cbstart%
Here is a paragraph that has (hypothetically) been changed since the last time it was sent to your advisor.
If the \texttt{changebars} boolean is set in the document's root, this paragraph will be identified by a change-bar.
\cbend%

\subsection{Related Work}%
\label{sec:ut:introduction:relatedWork}%
The Related Work section is always a good place to define some important terms.
\termdefinition{Important terms} are labels that you (or someone else) made up, possibly in an effort to elevate mundane concepts to a higher level of meaningfulness.

Of course, the Related Work~\citep{Robles:2001:Nature} section should also be rife with citations to the work of other researchers~\citep{Lloyd:2001:ICRA,Lederman:2004:HANDBOOK}.
Rife~\citep{Srinivasan:1996:ASME}.

In order to generate some nice cross-references in the bibliography(/ies), here are some more citations~\citep{Stein:1993:MergingOfSenses,HandbookOfMultisensory04}.
Here's a citation that you may recognize from earlier~\citep{Johansson:1996:HANDBRAIN}.

Be sure to note that a nice feature of the \texttt{natbib} package is its support for both \termdefinition{parenthetical citations} (like the ones used so far) and \termdefinition{textual citations} (where the authors of the work are typeset into the main text.
This is the kind of thing that \citet{Salcudean:1997:ASME} would probably have appreciated.
\citet{Hwang:2004:HAPSYMP} would have, too.

A respectable Related Work section should have a few~\citep{Hwang:2004:HAPSYMP} more~\citep{Kuchenbecker:2006:TVCG,Okamura:2001:ASME} citations~\citep{Lintern:1991:HUMANFACTORS}.
Definitely~\citep{Mertens:1981:AVIATIONSPACEENVMED, Wightman:1985:HUMANFACTORS}.

\subsection{Outline}%
This would be a good place to summarize the contributions of this chapter.

The remainder of this chapter is organized as follows.
In \autoref{sec:ut:subFig} there's a nice demonstration of a row of subfigures.
We then make a brief digression (due to page layout requirements) in \autoref{sec:ut:wrapFig} to show how small figures (or tables) can be nested within the body of the text.
Returning to the world of subfigures, there is an illustration of some trickery that you can do with blank figures and subfigures to create seemingly overlapping figures in \autoref{sec:ut:moreSubFig}.
\autoref{sec:ut:aligned} has some equations that use the \texttt{align} environment to line up nicely.
Some tall and narrow figures are shown in \autoref{sec:ut:tallFig}.
In \autoref{sec:ut:conclusions} we draw conclusions.
