%%%%%%%%%%%%%%%%%%%%%%%%%%%%%%%%%%%%%%%%%%%%%%%%%%%%%%%%%%%%%%%%%%%%%%%%%%%%%%%%
%% header.tex
%% 2009.01.09 - tedmunds
%%
%% This file is the header file used to perform chapter-specific configuration
%% for one chapter in the dissertation template.
%% This file is for preamble commands that are needed by the chapter (but not
%% generally needed by the document as a whole).
%% There is another file (./chapterHeader) that can be used to perform
%% non-preamble configuration at the start of the chapter.
%% Note that the commands issued (and packages used) here may not _conflict_
%% with the rest of the document; they should play nicely.
%% This file is included by the main header (../header/header.tex).
%%
%%%%%%%%%%%%%%%%%%%%%%%%%%%%%%%%%%%%%%%%%%%%%%%%%%%%%%%%%%%%%%%%%%%%%%%%%%%%%%%%

%% This chapter uses the graphicx package for figures.
\usepackage{graphicx}
%% This chapter needs some subfigures.
\usepackage{subfig}
%% This chapter uses the floatrow package to put captions beside tall figures.
\usepackage{floatrow}
%% This chapter uses the floatflt package to allow for small figures that are
%% nestled inside paragraphs.
\usepackage{floatflt}
%% The hhline package allows for double horizontal lines with appropriate
%% behaviour when crossing other lines (e.g., in tables).
\usepackage{hhline}
%% The array package allows for more control over the formatting of table
%% columns.
%\usepackage{array} 