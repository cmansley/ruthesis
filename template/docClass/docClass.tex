The \texttt{ruthesis} document class (\href{../ruthesis.cls}{\texttt{ruthesis.cls}}) is an extension of the \texttt{report} document class; it can be used to prepare dissertation manuscripts that comply with the \href{http://gsnb.rutgers.edu/guide.php3}{style requirements} of the Rutgers Graduate School.

\section{Contributors}
The \texttt{ruthesis} document class is the work of several contributors.
The original \texttt{ruthesis.cls} was created by Les Clowney.
It was based on the \LaTeX\ document style file, \texttt{ruthesis.sty}, to which both Dave Steiner and Tara Madhyastha contributed.
The document class was updated and combined with this dissertation template by Timothy Edmunds.

\section{Document Options}
The \texttt{ruthesis} document class defines the following options (in addition to the options provided by the report document class):
\begin{itemize}
\item \texttt{10pt}: typesets the manuscript with 10 pt font
\item \texttt{11pt}: typesets the manuscript with 11 pt font
\item \texttt{12pt}: typesets the manuscript with 12 pt font
\item \texttt{electronic}: specifies that the manuscript include features suitable for electronic publication (but not unsuitable for hard-copy).
    (In practice, this option merely sets the \texttt{electronic} boolean, so that the author can condition such things as the use of the \texttt{hyperref} package using $\backslash$\texttt{ifelec\-tronic}.)
\item \texttt{prelimdraft}: prepares a draft version of the manuscript (with a different title page, different headers and footers, etc.).
    Also sets the \texttt{prelimdraft} boolean so that the author can conditionally include material with $\backslash$\texttt{if\-prelimdraft}.
\item \texttt{nonsubmission}: prepares a version of the manuscript that omits final-submission features (such as the signature block).
    Also sets the \texttt{nonsub\-mission} boolean so that the author can conditionally include material with $\backslash$\texttt{if\-nonsubmission}.
\end{itemize}

\section{Booleans}
The \texttt{ruthesis} document class defines the following booleans on which the manuscript author can condition using the $\backslash$\texttt{if\it{boolean}}$\backslash$\texttt{else}$\backslash$fi \LaTeX\ construct:
\begin{itemize}
\item \texttt{electronic}: set by the \texttt{electronic} document class option.
      true if the manuscript is to be prepared in a method suitable for electronic distribution (e.g., uses the \texttt{hyperref} package).
\item \texttt{prelimdraft}: set by the \texttt{prelimdraft} document class option.
      true if the manuscript is to be prepared as a preliminary draft.
\item \texttt{nonsubmission}: set by the \texttt{nonsubmission} document class option.
      true if the manuscript is not the final submission version (with signature block).
\item \texttt{abstract}: set by the $\backslash$\texttt{abstract} command.
      Determines whether the abstract is typeset.
      May be set to false by the user after the $\backslash$\texttt{abstract} command in order to suppress the abstract.
\item \texttt{preface}: set by the $\backslash$\texttt{preface} command.
      Determines whether the preface is typeset.
      May be set to false by the user after the $\backslash$\texttt{preface} command in order to suppress the preface.
\item \texttt{acknowledgements}: set by the $\backslash$\texttt{acknowledgements} command.
      Determines whether the acknowledgements page is typeset.
      May be set to false by the user after the $\backslash$\texttt{acknowledgements} command in order to suppress the acknowledgements.
\item \texttt{dedication}: set by the $\backslash$\texttt{dedication} command.
      Determines whether the dedication is typeset.
      May be set to false by the user after the $\backslash$\texttt{dedication} command in order to suppress the dedication.
\item \texttt{abbreviationspage}: set by the $\backslash$\texttt{abbreviationspage} command.
      Determines whether the List of Abbreviations is typeset.
      May be set to false by the user after the $\backslash$\texttt{abbreviationspage} command in order to suppress the List of Abbreviations.
\end{itemize}

\section{Switch Commands}
Some of the commands provided by the ruthesis document class are one-way switches:
\begin{itemize}
\item $\backslash$\texttt{copyrightpage}: includes a copyright page when typesetting the manu\-script.
\item $\backslash$\texttt{figurespage}: includes a List of Figures when typesetting the manu\-script.
\item $\backslash$\texttt{tablespage}: includes a List of Tables when typesetting the manuscript.
\item $\backslash$\texttt{algorithmspage}: includes a List of Algorithms when typesetting the manuscript.
\item $\backslash$\texttt{phd}: sets the degree being sought to be a Ph.D.
\item $\backslash$\texttt{jointumdnj}: indicates that the degree is offered jointly with UMDNJ.
\end{itemize}

\section{Content Commands}
The ruthesis document class defines the following commands that the author should use to specify certain types of content:
\begin{itemize}
\item $\backslash$\texttt{title}\{{\it{The title of the dissertation}\}}
\item $\backslash$\texttt{author}\{{\it{The author of the dissertation}\}}
\item $\backslash$\texttt{degree}\{{\it{The degree being sought}\}}
      The default is Master of Science (or Doctor of Philosophy if $\backslash$\texttt{phd} is used.
\item $\backslash$\texttt{joint}\{{\it{The institution offering the joint degree}\}}
\item $\backslash$\texttt{director}\{{\it{The principal advisor of the author}\}}
\item $\backslash$\texttt{program}\{{\it{The degree program}\}}
\item $\backslash$\texttt{submissionyear}\{{\it{Year of submission to the Graduate School}\}}
      The default is the current year.
\item $\backslash$\texttt{submissionmonth}\{{\it{Month of submission to the Graduate School}\}}
      The default is the current month.
\item $\backslash$\texttt{approvals}\{{\it{The number of committee member signature lines}\}}
\item $\backslash$\texttt{abstract}\{{\it{The abstract of the dissertation}\}}
\item $\backslash$\texttt{acknowledgements}\{{\it{The acknowledgements}\}}
\item $\backslash$\texttt{dedication}\{{\it{The dedication}\}}
\item $\backslash$\texttt{preface}\{{\it{The preface}\}}
\item $\backslash$\texttt{abbreviationspage}\{{\it{The abbreviations used in the dissertation}\}}
\end{itemize}

\section{Environments}
The new environments defined by the ruthesis document class are:
\begin{itemize}
\item \texttt{bibliographysection}: this environment should be used to produce the required single-spacing of the bibliography.
\item \texttt{vita}: this environment should be used for the C.V.\ at the end of the manuscript.
\item \texttt{achievementlist}: this environment can be used to create the employment history and publication list in the C.V.
\end{itemize}

\section{Other Commands}
Some other commands are defined by the ruthesis document class:
\begin{itemize}
\item $\backslash$\texttt{beforepreface}: this command typesets all the material that comes before the preface (i.e., the title page, copyright page, abstract).
\item $\backslash$\texttt{afterpreface}: this command typesets all the material that comes between the \texttt{beforepreface} material and the body of the dissertation (i.e., the preface, the acknowledgements, the dedication, the table of contents, the lists of figures, tables and algorithms, etc.).
\item $\backslash$\texttt{appendix}: this command signals that all subsequent chapters are appendices.
\end{itemize}

\section{Margins}
The \href{http://gsnb.rutgers.edu/guide.php3}{Graduate School Style Guide} specifies that left margins must be 1.5 inches, and that the top, right, and bottom margins must be 1 inch.
It is assumed that these margins apply to all pages in the manuscript.
Since \LaTeX\ is not always successful in complying with the specified margins (due to overfull boxes), the margins defined in the ruthesis document class are:
\begin{itemize}
\item Left: 108 pt (1.5")
\item Right: 92 pt (\ensuremath{1\frac{5}{18}"})
\item Top: 72 pt (1")
\item Bottom: 92 pt (\ensuremath{1\frac{5}{18}"})
\end{itemize}

\section{Single-Sided}
The style guide never explicitly specifies whether the manuscript should be single- or double-sided, but the fact that the left margin is larger than the right margin on all pages (not just odd-numbered pages) suggests that all pages are to be bound on the left.
Therefore, the assumption is that pages should be single-sided --- with electronic submission, this only really affects the hard-copy of the title page and abstract that need to handed in.