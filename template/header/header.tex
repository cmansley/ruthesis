%%%%%%%%%%%%%%%%%%%%%%%%%%%%%%%%%%%%%%%%%%%%%%%%%%%%%%%%%%%%%%%%%%%%%%%%%%%%%%%%
%% header.tex
%% 2009.01.08 - tedmunds
%%
%% This file is the header file used to specify document-wide configuration for
%% the dissertation template file.
%% This file is included by the template's root file (../dissertation.tex).
%%
%%%%%%%%%%%%%%%%%%%%%%%%%%%%%%%%%%%%%%%%%%%%%%%%%%%%%%%%%%%%%%%%%%%%%%%%%%%%%%%%

%% The Graduate School Style Guide doesn't specify a typeface to use, only that
%% it should be easy-to-read.
%% The mathptmx package changes the default roman font family to Times, and uses
%% the virtual mathptmx fonts for math.
%\usepackage{mathptmx}
%% The mathpazo package changes the default roman font family to Adobe
%% Palatino, and uses the virtual mathpazo fonts for math.
\usepackage{mathpazo}
%% The times package changes the default roman font family to Times, but leaves
%% the default math font as Computer Modern
%\usepackage{times}
%% The helvet package changes the sans serif font family to Helvetica.
\usepackage{helvet}

%% The ifpdf package makes available the pdf boolean that (supposedly)
%% indicates whether the document is being compiled by pdfLaTeX (or some
%% equivalent).  This lets the author condition such things as the graphics
%% extensions, so that the source can be compiled by plain LaTeX as well as
%% more pdfish variants.
\usepackage{ifpdf}

%% The import package simplifies a tree-like document source code structure.
%% When the \include and \input commands load files, any nested \input or
%% \include (or \includegraphics) commands are treated as though they were
%% issued at the document root level.
%% With the \import command, subsequently included files are loaded relative
%% to the imported directory.
\usepackage{import}

%% Everybody needs a little math formatting from the good people at the
%% American Mathematical Society
\usepackage{amsmath}

%% The \tablespage command induces the generation of the List of Tables page
\tablespage
%% The \figurespage command induces the generation of the List of Figures page
\figurespage
%% The \algorithmspage command induces the generation of the List of Algorithms
%% page
%\algorithmspage

%% Based on the numericalcitation switch (specified in the root document), we
%% configure the natbib package to either create numerical citations or
%% the richer author-year format citations.
\ifnumericalcitation
%% Use numerical citations, sort the references in ascending order within the
%% citation, and use square brackets around citations.
%% Note that although natbib provides the sort&compress option to compress
%% numerical citation lists (like [3-6] instead of [3,4,5,6]), that option is
%% less friendly with electronic versions, since the compressed citations don't
%% get links.
%% So unless you are sure that you don't want to produce any electronic
%% versions (or you don't worry about them subtley changing length), steer
%% clear of that option.
\usepackage[numbers,sort,square]{natbib}
\else
%% Use square braces around citations.
\usepackage[square]{natbib}
\fi
%% Compilation sometimes hangs (and locks the log file) if bibunits is included
%% before natbib (though seemingly only when the numbers style is used for
%% natbib).
\ifperchapterbib
  \ifendbib
    %% If we are producing per-chapter bibliographies _and_ a final
    %% bibliography, we want the citations in each chapter to be automatically
    %% copied into the citation list for the final bibliography.
    \usepackage[globalcitecopy]{bibunits}
  \else
    %% If we are only having bibliographies at the end of each chapter, no
    %% arguments are required.
    \usepackage{bibunits}
  \fi% end ifendbib
  %% If we are having end-of-chapter bibliographies, we need to redefine the
  %% \bibsection command.
  %% The bibliography will be an unnumbered section titled "References", and
  %% the section will be included in the table of contents.
  \def\bibsection{\pagebreak[4]\section*{References}\addcontentsline{toc}{section}{References}}
\else
  \usepackage{bibunits}
  %% The bibliography will be an unnumbered chapter titled "Bibliography".
  \def\bibsection{\unnumberedchapter{Bibliography}}
\fi% end ifperchapterbib
%% There are some bugs in the bibunits package that need to be patched.
\usepackage{patches/bibunitspatch}

%% The Graduate School Style Guide is not clear/definitive about the citation
%% style to be used.  But is does seem that if you use numerical citations,
%% they want them to be in ascending order (i.e., sorted by citation order, not
%% alphabetically.
\ifnumericalcitation
  %% If we are using numerical citation (set in the root document), we need to
  %% use the ruthesisciteorder BibTeX style file.
  \ifperchapterbib
    %% If we are including a bibliography at the end of each chapter, the
    %% command to set the BibTeX style for all of the bibliographies is
    %% \defaultbibliographystyle.
    \defaultbibliographystyle{../bibtexery/ruthesisciteorder}
    %% If there is a bibliography for each chapter as well, the global
    %% bibliography should be numberless and in alphabetical order
    \bibliographystyle{../bibtexery/ruthesis}
  \else
    %% If we don't have a bibliography at the end of each chapter, the
    %% command to set the BibTeX style is the standard \bibliographystyle.
    \bibliographystyle{../bibtexery/ruthesisciteorder}
  \fi
\else
  %% If we are not using numerical citations, we can use the ruthesis BibTeX
  %% style file that allows for nice author-year style citations.
  \bibliographystyle{../bibtexery/ruthesis}
  \ifperchapterbib
    %% If we are including a bibliography at the end of each chapter, the
    %% command to set the BibTeX style for all of the bibliographies is
    %% \defaultbibliographystyle.
    \defaultbibliographystyle{../bibtexery/ruthesis}
  \else
  \fi
\fi

%% We define commands for the text of the dissertation's title and author,
%% since they will be used in a couple of places.
\newcommand{\titletext}{We Learned \LaTeX\ So You Don't Have To: A Self-Documenting Template for Preparing Dissertations for the Rutgers Graduate School}
%% If you use any special formatting that won't work in plain text (for the
%% PDF document information), create a safe version as well.
\newcommand{\safetitletext}{We Learned LaTeX So You Don't Have To: A Self-Documenting Template for Preparing Dissertations for the Rutgers Graduate School}
\providecommand{\safetitletext}{\titletext}
\newcommand{\authortext}{Timothy Edmunds}

%% A handy way of formatting the author's notes-to-self (or -to-advisor) is
%% to separate them from the surrounding text with an icon, and to insert an
%% icon in the margin to make the note obvious.
%% When the document is being compiled for an audience that doesn't need the
%% notes, they can all be turned off with the notes boolean (set in the root
%% document), rather than having to do search-and-delete (or -and-comment).
\ifnotes
\providecommand{\note}[1]
    {\marginpar{\LARGE $\spadesuit$}$\spadesuit$ {\bf #1} $\spadesuit$}
\else
%% If notes are turned off, just define an empty notes command.
\providecommand{\note}[1]
    {}
\fi % notes

%% In addition to this header file, each chapter can define a header with
%% commands/packages specific to that chapter.  That way, when a chapter is
%% not being typeset, the header can be ommitted to simplify things.
\ifUsefulTechniques
%%%%%%%%%%%%%%%%%%%%%%%%%%%%%%%%%%%%%%%%%%%%%%%%%%%%%%%%%%%%%%%%%%%%%%%%%%%%%%%%
%% header.tex
%% 2009.01.08 - tedmunds
%%
%% This file is the header file used to specify document-wide configuration for
%% the dissertation template file.
%% This file is included by the template's root file (../dissertation.tex).
%%
%%%%%%%%%%%%%%%%%%%%%%%%%%%%%%%%%%%%%%%%%%%%%%%%%%%%%%%%%%%%%%%%%%%%%%%%%%%%%%%%

%% The Graduate School Style Guide doesn't specify a typeface to use, only that
%% it should be easy-to-read.
%% The mathptmx package changes the default roman font family to Times, and uses
%% the virtual mathptmx fonts for math.
%\usepackage{mathptmx}
%% The mathpazo package changes the default roman font family to Adobe
%% Palatino, and uses the virtual mathpazo fonts for math.
\usepackage{mathpazo}
%% The times package changes the default roman font family to Times, but leaves
%% the default math font as Computer Modern
%\usepackage{times}
%% The helvet package changes the sans serif font family to Helvetica.
\usepackage{helvet}

%% The ifpdf package makes available the pdf boolean that (supposedly)
%% indicates whether the document is being compiled by pdfLaTeX (or some
%% equivalent).  This lets the author condition such things as the graphics
%% extensions, so that the source can be compiled by plain LaTeX as well as
%% more pdfish variants.
\usepackage{ifpdf}

%% The import package simplifies a tree-like document source code structure.
%% When the \include and \input commands load files, any nested \input or
%% \include (or \includegraphics) commands are treated as though they were
%% issued at the document root level.
%% With the \import command, subsequently included files are loaded relative
%% to the imported directory.
\usepackage{import}

%% Everybody needs a little math formatting from the good people at the
%% American Mathematical Society
\usepackage{amsmath}

%% The \tablespage command induces the generation of the List of Tables page
\tablespage
%% The \figurespage command induces the generation of the List of Figures page
\figurespage
%% The \algorithmspage command induces the generation of the List of Algorithms
%% page
%\algorithmspage

%% Based on the numericalcitation switch (specified in the root document), we
%% configure the natbib package to either create numerical citations or
%% the richer author-year format citations.
\ifnumericalcitation
%% Use numerical citations, sort the references in ascending order within the
%% citation, and use square brackets around citations.
%% Note that although natbib provides the sort&compress option to compress
%% numerical citation lists (like [3-6] instead of [3,4,5,6]), that option is
%% less friendly with electronic versions, since the compressed citations don't
%% get links.
%% So unless you are sure that you don't want to produce any electronic
%% versions (or you don't worry about them subtley changing length), steer
%% clear of that option.
\usepackage[numbers,sort,square]{natbib}
\else
%% Use square braces around citations.
\usepackage[square]{natbib}
\fi
%% Compilation sometimes hangs (and locks the log file) if bibunits is included
%% before natbib (though seemingly only when the numbers style is used for
%% natbib).
\ifperchapterbib
  \ifendbib
    %% If we are producing per-chapter bibliographies _and_ a final
    %% bibliography, we want the citations in each chapter to be automatically
    %% copied into the citation list for the final bibliography.
    \usepackage[globalcitecopy]{bibunits}
  \else
    %% If we are only having bibliographies at the end of each chapter, no
    %% arguments are required.
    \usepackage{bibunits}
  \fi% end ifendbib
  %% If we are having end-of-chapter bibliographies, we need to redefine the
  %% \bibsection command.
  %% The bibliography will be an unnumbered section titled "References", and
  %% the section will be included in the table of contents.
  \def\bibsection{\pagebreak[4]\section*{References}\addcontentsline{toc}{section}{References}}
\else
  \usepackage{bibunits}
  %% The bibliography will be an unnumbered chapter titled "Bibliography".
  \def\bibsection{\unnumberedchapter{Bibliography}}
\fi% end ifperchapterbib
%% There are some bugs in the bibunits package that need to be patched.
\usepackage{patches/bibunitspatch}

%% The Graduate School Style Guide is not clear/definitive about the citation
%% style to be used.  But is does seem that if you use numerical citations,
%% they want them to be in ascending order (i.e., sorted by citation order, not
%% alphabetically.
\ifnumericalcitation
  %% If we are using numerical citation (set in the root document), we need to
  %% use the ruthesisciteorder BibTeX style file.
  \ifperchapterbib
    %% If we are including a bibliography at the end of each chapter, the
    %% command to set the BibTeX style for all of the bibliographies is
    %% \defaultbibliographystyle.
    \defaultbibliographystyle{../bibtexery/ruthesisciteorder}
    %% If there is a bibliography for each chapter as well, the global
    %% bibliography should be numberless and in alphabetical order
    \bibliographystyle{../bibtexery/ruthesis}
  \else
    %% If we don't have a bibliography at the end of each chapter, the
    %% command to set the BibTeX style is the standard \bibliographystyle.
    \bibliographystyle{../bibtexery/ruthesisciteorder}
  \fi
\else
  %% If we are not using numerical citations, we can use the ruthesis BibTeX
  %% style file that allows for nice author-year style citations.
  \bibliographystyle{../bibtexery/ruthesis}
  \ifperchapterbib
    %% If we are including a bibliography at the end of each chapter, the
    %% command to set the BibTeX style for all of the bibliographies is
    %% \defaultbibliographystyle.
    \defaultbibliographystyle{../bibtexery/ruthesis}
  \else
  \fi
\fi

%% We define commands for the text of the dissertation's title and author,
%% since they will be used in a couple of places.
\newcommand{\titletext}{We Learned \LaTeX\ So You Don't Have To: A Self-Documenting Template for Preparing Dissertations for the Rutgers Graduate School}
%% If you use any special formatting that won't work in plain text (for the
%% PDF document information), create a safe version as well.
\newcommand{\safetitletext}{We Learned LaTeX So You Don't Have To: A Self-Documenting Template for Preparing Dissertations for the Rutgers Graduate School}
\providecommand{\safetitletext}{\titletext}
\newcommand{\authortext}{Timothy Edmunds}

%% A handy way of formatting the author's notes-to-self (or -to-advisor) is
%% to separate them from the surrounding text with an icon, and to insert an
%% icon in the margin to make the note obvious.
%% When the document is being compiled for an audience that doesn't need the
%% notes, they can all be turned off with the notes boolean (set in the root
%% document), rather than having to do search-and-delete (or -and-comment).
\ifnotes
\providecommand{\note}[1]
    {\marginpar{\LARGE $\spadesuit$}$\spadesuit$ {\bf #1} $\spadesuit$}
\else
%% If notes are turned off, just define an empty notes command.
\providecommand{\note}[1]
    {}
\fi % notes

%% In addition to this header file, each chapter can define a header with
%% commands/packages specific to that chapter.  That way, when a chapter is
%% not being typeset, the header can be ommitted to simplify things.
\ifUsefulTechniques
%%%%%%%%%%%%%%%%%%%%%%%%%%%%%%%%%%%%%%%%%%%%%%%%%%%%%%%%%%%%%%%%%%%%%%%%%%%%%%%%
%% header.tex
%% 2009.01.08 - tedmunds
%%
%% This file is the header file used to specify document-wide configuration for
%% the dissertation template file.
%% This file is included by the template's root file (../dissertation.tex).
%%
%%%%%%%%%%%%%%%%%%%%%%%%%%%%%%%%%%%%%%%%%%%%%%%%%%%%%%%%%%%%%%%%%%%%%%%%%%%%%%%%

%% The Graduate School Style Guide doesn't specify a typeface to use, only that
%% it should be easy-to-read.
%% The mathptmx package changes the default roman font family to Times, and uses
%% the virtual mathptmx fonts for math.
%\usepackage{mathptmx}
%% The mathpazo package changes the default roman font family to Adobe
%% Palatino, and uses the virtual mathpazo fonts for math.
\usepackage{mathpazo}
%% The times package changes the default roman font family to Times, but leaves
%% the default math font as Computer Modern
%\usepackage{times}
%% The helvet package changes the sans serif font family to Helvetica.
\usepackage{helvet}

%% The ifpdf package makes available the pdf boolean that (supposedly)
%% indicates whether the document is being compiled by pdfLaTeX (or some
%% equivalent).  This lets the author condition such things as the graphics
%% extensions, so that the source can be compiled by plain LaTeX as well as
%% more pdfish variants.
\usepackage{ifpdf}

%% The import package simplifies a tree-like document source code structure.
%% When the \include and \input commands load files, any nested \input or
%% \include (or \includegraphics) commands are treated as though they were
%% issued at the document root level.
%% With the \import command, subsequently included files are loaded relative
%% to the imported directory.
\usepackage{import}

%% Everybody needs a little math formatting from the good people at the
%% American Mathematical Society
\usepackage{amsmath}

%% The \tablespage command induces the generation of the List of Tables page
\tablespage
%% The \figurespage command induces the generation of the List of Figures page
\figurespage
%% The \algorithmspage command induces the generation of the List of Algorithms
%% page
%\algorithmspage

%% Based on the numericalcitation switch (specified in the root document), we
%% configure the natbib package to either create numerical citations or
%% the richer author-year format citations.
\ifnumericalcitation
%% Use numerical citations, sort the references in ascending order within the
%% citation, and use square brackets around citations.
%% Note that although natbib provides the sort&compress option to compress
%% numerical citation lists (like [3-6] instead of [3,4,5,6]), that option is
%% less friendly with electronic versions, since the compressed citations don't
%% get links.
%% So unless you are sure that you don't want to produce any electronic
%% versions (or you don't worry about them subtley changing length), steer
%% clear of that option.
\usepackage[numbers,sort,square]{natbib}
\else
%% Use square braces around citations.
\usepackage[square]{natbib}
\fi
%% Compilation sometimes hangs (and locks the log file) if bibunits is included
%% before natbib (though seemingly only when the numbers style is used for
%% natbib).
\ifperchapterbib
  \ifendbib
    %% If we are producing per-chapter bibliographies _and_ a final
    %% bibliography, we want the citations in each chapter to be automatically
    %% copied into the citation list for the final bibliography.
    \usepackage[globalcitecopy]{bibunits}
  \else
    %% If we are only having bibliographies at the end of each chapter, no
    %% arguments are required.
    \usepackage{bibunits}
  \fi% end ifendbib
  %% If we are having end-of-chapter bibliographies, we need to redefine the
  %% \bibsection command.
  %% The bibliography will be an unnumbered section titled "References", and
  %% the section will be included in the table of contents.
  \def\bibsection{\pagebreak[4]\section*{References}\addcontentsline{toc}{section}{References}}
\else
  \usepackage{bibunits}
  %% The bibliography will be an unnumbered chapter titled "Bibliography".
  \def\bibsection{\unnumberedchapter{Bibliography}}
\fi% end ifperchapterbib
%% There are some bugs in the bibunits package that need to be patched.
\usepackage{patches/bibunitspatch}

%% The Graduate School Style Guide is not clear/definitive about the citation
%% style to be used.  But is does seem that if you use numerical citations,
%% they want them to be in ascending order (i.e., sorted by citation order, not
%% alphabetically.
\ifnumericalcitation
  %% If we are using numerical citation (set in the root document), we need to
  %% use the ruthesisciteorder BibTeX style file.
  \ifperchapterbib
    %% If we are including a bibliography at the end of each chapter, the
    %% command to set the BibTeX style for all of the bibliographies is
    %% \defaultbibliographystyle.
    \defaultbibliographystyle{../bibtexery/ruthesisciteorder}
    %% If there is a bibliography for each chapter as well, the global
    %% bibliography should be numberless and in alphabetical order
    \bibliographystyle{../bibtexery/ruthesis}
  \else
    %% If we don't have a bibliography at the end of each chapter, the
    %% command to set the BibTeX style is the standard \bibliographystyle.
    \bibliographystyle{../bibtexery/ruthesisciteorder}
  \fi
\else
  %% If we are not using numerical citations, we can use the ruthesis BibTeX
  %% style file that allows for nice author-year style citations.
  \bibliographystyle{../bibtexery/ruthesis}
  \ifperchapterbib
    %% If we are including a bibliography at the end of each chapter, the
    %% command to set the BibTeX style for all of the bibliographies is
    %% \defaultbibliographystyle.
    \defaultbibliographystyle{../bibtexery/ruthesis}
  \else
  \fi
\fi

%% We define commands for the text of the dissertation's title and author,
%% since they will be used in a couple of places.
\newcommand{\titletext}{We Learned \LaTeX\ So You Don't Have To: A Self-Documenting Template for Preparing Dissertations for the Rutgers Graduate School}
%% If you use any special formatting that won't work in plain text (for the
%% PDF document information), create a safe version as well.
\newcommand{\safetitletext}{We Learned LaTeX So You Don't Have To: A Self-Documenting Template for Preparing Dissertations for the Rutgers Graduate School}
\providecommand{\safetitletext}{\titletext}
\newcommand{\authortext}{Timothy Edmunds}

%% A handy way of formatting the author's notes-to-self (or -to-advisor) is
%% to separate them from the surrounding text with an icon, and to insert an
%% icon in the margin to make the note obvious.
%% When the document is being compiled for an audience that doesn't need the
%% notes, they can all be turned off with the notes boolean (set in the root
%% document), rather than having to do search-and-delete (or -and-comment).
\ifnotes
\providecommand{\note}[1]
    {\marginpar{\LARGE $\spadesuit$}$\spadesuit$ {\bf #1} $\spadesuit$}
\else
%% If notes are turned off, just define an empty notes command.
\providecommand{\note}[1]
    {}
\fi % notes

%% In addition to this header file, each chapter can define a header with
%% commands/packages specific to that chapter.  That way, when a chapter is
%% not being typeset, the header can be ommitted to simplify things.
\ifUsefulTechniques
%%%%%%%%%%%%%%%%%%%%%%%%%%%%%%%%%%%%%%%%%%%%%%%%%%%%%%%%%%%%%%%%%%%%%%%%%%%%%%%%
%% header.tex
%% 2009.01.08 - tedmunds
%%
%% This file is the header file used to specify document-wide configuration for
%% the dissertation template file.
%% This file is included by the template's root file (../dissertation.tex).
%%
%%%%%%%%%%%%%%%%%%%%%%%%%%%%%%%%%%%%%%%%%%%%%%%%%%%%%%%%%%%%%%%%%%%%%%%%%%%%%%%%

%% The Graduate School Style Guide doesn't specify a typeface to use, only that
%% it should be easy-to-read.
%% The mathptmx package changes the default roman font family to Times, and uses
%% the virtual mathptmx fonts for math.
%\usepackage{mathptmx}
%% The mathpazo package changes the default roman font family to Adobe
%% Palatino, and uses the virtual mathpazo fonts for math.
\usepackage{mathpazo}
%% The times package changes the default roman font family to Times, but leaves
%% the default math font as Computer Modern
%\usepackage{times}
%% The helvet package changes the sans serif font family to Helvetica.
\usepackage{helvet}

%% The ifpdf package makes available the pdf boolean that (supposedly)
%% indicates whether the document is being compiled by pdfLaTeX (or some
%% equivalent).  This lets the author condition such things as the graphics
%% extensions, so that the source can be compiled by plain LaTeX as well as
%% more pdfish variants.
\usepackage{ifpdf}

%% The import package simplifies a tree-like document source code structure.
%% When the \include and \input commands load files, any nested \input or
%% \include (or \includegraphics) commands are treated as though they were
%% issued at the document root level.
%% With the \import command, subsequently included files are loaded relative
%% to the imported directory.
\usepackage{import}

%% Everybody needs a little math formatting from the good people at the
%% American Mathematical Society
\usepackage{amsmath}

%% The \tablespage command induces the generation of the List of Tables page
\tablespage
%% The \figurespage command induces the generation of the List of Figures page
\figurespage
%% The \algorithmspage command induces the generation of the List of Algorithms
%% page
%\algorithmspage

%% Based on the numericalcitation switch (specified in the root document), we
%% configure the natbib package to either create numerical citations or
%% the richer author-year format citations.
\ifnumericalcitation
%% Use numerical citations, sort the references in ascending order within the
%% citation, and use square brackets around citations.
%% Note that although natbib provides the sort&compress option to compress
%% numerical citation lists (like [3-6] instead of [3,4,5,6]), that option is
%% less friendly with electronic versions, since the compressed citations don't
%% get links.
%% So unless you are sure that you don't want to produce any electronic
%% versions (or you don't worry about them subtley changing length), steer
%% clear of that option.
\usepackage[numbers,sort,square]{natbib}
\else
%% Use square braces around citations.
\usepackage[square]{natbib}
\fi
%% Compilation sometimes hangs (and locks the log file) if bibunits is included
%% before natbib (though seemingly only when the numbers style is used for
%% natbib).
\ifperchapterbib
  \ifendbib
    %% If we are producing per-chapter bibliographies _and_ a final
    %% bibliography, we want the citations in each chapter to be automatically
    %% copied into the citation list for the final bibliography.
    \usepackage[globalcitecopy]{bibunits}
  \else
    %% If we are only having bibliographies at the end of each chapter, no
    %% arguments are required.
    \usepackage{bibunits}
  \fi% end ifendbib
  %% If we are having end-of-chapter bibliographies, we need to redefine the
  %% \bibsection command.
  %% The bibliography will be an unnumbered section titled "References", and
  %% the section will be included in the table of contents.
  \def\bibsection{\pagebreak[4]\section*{References}\addcontentsline{toc}{section}{References}}
\else
  \usepackage{bibunits}
  %% The bibliography will be an unnumbered chapter titled "Bibliography".
  \def\bibsection{\unnumberedchapter{Bibliography}}
\fi% end ifperchapterbib
%% There are some bugs in the bibunits package that need to be patched.
\usepackage{patches/bibunitspatch}

%% The Graduate School Style Guide is not clear/definitive about the citation
%% style to be used.  But is does seem that if you use numerical citations,
%% they want them to be in ascending order (i.e., sorted by citation order, not
%% alphabetically.
\ifnumericalcitation
  %% If we are using numerical citation (set in the root document), we need to
  %% use the ruthesisciteorder BibTeX style file.
  \ifperchapterbib
    %% If we are including a bibliography at the end of each chapter, the
    %% command to set the BibTeX style for all of the bibliographies is
    %% \defaultbibliographystyle.
    \defaultbibliographystyle{../bibtexery/ruthesisciteorder}
    %% If there is a bibliography for each chapter as well, the global
    %% bibliography should be numberless and in alphabetical order
    \bibliographystyle{../bibtexery/ruthesis}
  \else
    %% If we don't have a bibliography at the end of each chapter, the
    %% command to set the BibTeX style is the standard \bibliographystyle.
    \bibliographystyle{../bibtexery/ruthesisciteorder}
  \fi
\else
  %% If we are not using numerical citations, we can use the ruthesis BibTeX
  %% style file that allows for nice author-year style citations.
  \bibliographystyle{../bibtexery/ruthesis}
  \ifperchapterbib
    %% If we are including a bibliography at the end of each chapter, the
    %% command to set the BibTeX style for all of the bibliographies is
    %% \defaultbibliographystyle.
    \defaultbibliographystyle{../bibtexery/ruthesis}
  \else
  \fi
\fi

%% We define commands for the text of the dissertation's title and author,
%% since they will be used in a couple of places.
\newcommand{\titletext}{We Learned \LaTeX\ So You Don't Have To: A Self-Documenting Template for Preparing Dissertations for the Rutgers Graduate School}
%% If you use any special formatting that won't work in plain text (for the
%% PDF document information), create a safe version as well.
\newcommand{\safetitletext}{We Learned LaTeX So You Don't Have To: A Self-Documenting Template for Preparing Dissertations for the Rutgers Graduate School}
\providecommand{\safetitletext}{\titletext}
\newcommand{\authortext}{Timothy Edmunds}

%% A handy way of formatting the author's notes-to-self (or -to-advisor) is
%% to separate them from the surrounding text with an icon, and to insert an
%% icon in the margin to make the note obvious.
%% When the document is being compiled for an audience that doesn't need the
%% notes, they can all be turned off with the notes boolean (set in the root
%% document), rather than having to do search-and-delete (or -and-comment).
\ifnotes
\providecommand{\note}[1]
    {\marginpar{\LARGE $\spadesuit$}$\spadesuit$ {\bf #1} $\spadesuit$}
\else
%% If notes are turned off, just define an empty notes command.
\providecommand{\note}[1]
    {}
\fi % notes

%% In addition to this header file, each chapter can define a header with
%% commands/packages specific to that chapter.  That way, when a chapter is
%% not being typeset, the header can be ommitted to simplify things.
\ifUsefulTechniques
\input{usefulTechniques/header}
\else \fi
\ifAppendices
\input{appendices/manySubfigsHeader}
\else \fi

%% When iterating with an advisor or other reader, it may be helpful to
%% highlight content changes with change-bars.  This can be accomplished
%% using the changebar package.  When preparing the document for other
%% audiences, the change-bars can be turned off (in the document root file)
%% by controlling the changebars boolean.
\ifchangebars%
\usepackage{changebar}
\else%
%% If change-bars are off, define empty start/end commands.
\providecommand{\cbstart}{}%
\providecommand{\cbend}{}%
\fi% end ifchangebars

%% hyperref must (allegedly) be the _last_ package included
\ifelectronic
  %% If the dissertation is being typeset in an electronic-friendly way
  %% (controlled by the electronic document class option), we'll use
  %% the hyperref package to create a nicely linked pdf (with clickable
  %% citations, pdf bookmarks, etc.).
  \usepackage[verbose,pagebackref, pdftitle={\safetitletext}, pdfauthor={\authortext}, pdfsubject={Dissertation}]{hyperref}
  \ifpdf\else
  %% If compling with the LaTeX->dvi->ps->pdf pipeline, the dvips driver
  %% does not allow link text to wrap at the end of a line (even if you
  %% add \- hypenation instructions).
  %% This can cause lots of margin problems.
  %% The breakurl package addresses this problem, but only for \url links,
  %% not all the other types of links (like internal document links).
  %\usepackage{breakurl}
  %% A workaround is to use the breaklinks option of the hyperref package.
  %% That option is supposed to be for internal use, but it can be used to
  %% force wrapping; unfortunately, when viewed with a PDF viewer (rather
  %% than printed), the active link area will be misplaced.
  %% If good links are important to you, you might want to consider the
  %% pdflatex route.
  \hypersetup{breaklinks}
  \fi% end ifpdf
  %% In its capacity as a template, this document has some relative links
  %% to other files.
  %% To make relative file links work, we specify an empty prefix.
  \hypersetup{linkfileprefix={}}
\else
  %% If the electronic option is turned off, we'll still use the hyperref
  %% package, but we'll set it to draft mode.
  \usepackage[draft]{hyperref}
\fi % electronic
%% There some bugs in the backref package (which is used by the hyperref
%% package to provide links back from a reference to the citation location),
%% so we'll include a patch to work around the problems.
\usepackage{patches/backrefpatch}
%% The hyperref package provides the \autoref command that figures out what
%% type of entity you are creating a reference to, and makes the name of that
%% type of entity part of the reference (and part of the hyperlink).
%% Here we define the names that we want to be used for certain types of
%% entities.
\def\chapterautorefname{{C}hapter}
\def\sectionautorefname{{S}ection}
\def\subsectionautorefname{\sectionautorefname}
\newcommand{\subfigureautorefname}{\figureautorefname}

%% Import the file of author-defined common macros
\import{header/}{commands} 
\else \fi
\ifAppendices
%%%%%%%%%%%%%%%%%%%%%%%%%%%%%%%%%%%%%%%%%%%%%%%%%%%%%%%%%%%%%%%%%%%%%%%%%%%%%%%%
%% manySubfigsheader.tex
%% 2009.01.09 - tedmunds
%%
%% This file is the header file used to perform chapter-specific configuration
%% for one chapter in the dissertation template.
%% This file is for preamble commands that are needed by the chapter (but not
%% generally needed by the document as a whole).
%% There is another file (./manySubfigsChapterHeader) that can be used to
%% perform non-preamble configuration at the start of the chapter.
%% Note that the commands issued (and packages used) here may not _conflict_
%% with the rest of the document; they should play nicely.
%% This file is included by the main header (../header/header.tex).
%%
%%%%%%%%%%%%%%%%%%%%%%%%%%%%%%%%%%%%%%%%%%%%%%%%%%%%%%%%%%%%%%%%%%%%%%%%%%%%%%%%

%% This chapter uses the graphicx package for figures.
\usepackage{graphicx}
%% This chapter uses the subfig package to break a figure into parts (on
%% separate pages) while maintaining the sub-figure labels.
\usepackage{subfig}
%% The alphalph package allows us to have more than 26 things in an alphabetically numbered
%% list
\usepackage{alphalph}

%% Replace the default \alph command with \alphalph (from the alphalph package) so that
%% \alph can handle values higher than 26.  This is useful in situations where you have
%% more than 26 subfigures in a figure.  Or more than 26 appendices.
\renewcommand{\alph}[1]{\alphalph{\value{#1}}} 
\else \fi

%% When iterating with an advisor or other reader, it may be helpful to
%% highlight content changes with change-bars.  This can be accomplished
%% using the changebar package.  When preparing the document for other
%% audiences, the change-bars can be turned off (in the document root file)
%% by controlling the changebars boolean.
\ifchangebars%
\usepackage{changebar}
\else%
%% If change-bars are off, define empty start/end commands.
\providecommand{\cbstart}{}%
\providecommand{\cbend}{}%
\fi% end ifchangebars

%% hyperref must (allegedly) be the _last_ package included
\ifelectronic
  %% If the dissertation is being typeset in an electronic-friendly way
  %% (controlled by the electronic document class option), we'll use
  %% the hyperref package to create a nicely linked pdf (with clickable
  %% citations, pdf bookmarks, etc.).
  \usepackage[verbose,pagebackref, pdftitle={\safetitletext}, pdfauthor={\authortext}, pdfsubject={Dissertation}]{hyperref}
  \ifpdf\else
  %% If compling with the LaTeX->dvi->ps->pdf pipeline, the dvips driver
  %% does not allow link text to wrap at the end of a line (even if you
  %% add \- hypenation instructions).
  %% This can cause lots of margin problems.
  %% The breakurl package addresses this problem, but only for \url links,
  %% not all the other types of links (like internal document links).
  %\usepackage{breakurl}
  %% A workaround is to use the breaklinks option of the hyperref package.
  %% That option is supposed to be for internal use, but it can be used to
  %% force wrapping; unfortunately, when viewed with a PDF viewer (rather
  %% than printed), the active link area will be misplaced.
  %% If good links are important to you, you might want to consider the
  %% pdflatex route.
  \hypersetup{breaklinks}
  \fi% end ifpdf
  %% In its capacity as a template, this document has some relative links
  %% to other files.
  %% To make relative file links work, we specify an empty prefix.
  \hypersetup{linkfileprefix={}}
\else
  %% If the electronic option is turned off, we'll still use the hyperref
  %% package, but we'll set it to draft mode.
  \usepackage[draft]{hyperref}
\fi % electronic
%% There some bugs in the backref package (which is used by the hyperref
%% package to provide links back from a reference to the citation location),
%% so we'll include a patch to work around the problems.
\usepackage{patches/backrefpatch}
%% The hyperref package provides the \autoref command that figures out what
%% type of entity you are creating a reference to, and makes the name of that
%% type of entity part of the reference (and part of the hyperlink).
%% Here we define the names that we want to be used for certain types of
%% entities.
\def\chapterautorefname{{C}hapter}
\def\sectionautorefname{{S}ection}
\def\subsectionautorefname{\sectionautorefname}
\newcommand{\subfigureautorefname}{\figureautorefname}

%% Import the file of author-defined common macros
\import{header/}{commands} 
\else \fi
\ifAppendices
%%%%%%%%%%%%%%%%%%%%%%%%%%%%%%%%%%%%%%%%%%%%%%%%%%%%%%%%%%%%%%%%%%%%%%%%%%%%%%%%
%% manySubfigsheader.tex
%% 2009.01.09 - tedmunds
%%
%% This file is the header file used to perform chapter-specific configuration
%% for one chapter in the dissertation template.
%% This file is for preamble commands that are needed by the chapter (but not
%% generally needed by the document as a whole).
%% There is another file (./manySubfigsChapterHeader) that can be used to
%% perform non-preamble configuration at the start of the chapter.
%% Note that the commands issued (and packages used) here may not _conflict_
%% with the rest of the document; they should play nicely.
%% This file is included by the main header (../header/header.tex).
%%
%%%%%%%%%%%%%%%%%%%%%%%%%%%%%%%%%%%%%%%%%%%%%%%%%%%%%%%%%%%%%%%%%%%%%%%%%%%%%%%%

%% This chapter uses the graphicx package for figures.
\usepackage{graphicx}
%% This chapter uses the subfig package to break a figure into parts (on
%% separate pages) while maintaining the sub-figure labels.
\usepackage{subfig}
%% The alphalph package allows us to have more than 26 things in an alphabetically numbered
%% list
\usepackage{alphalph}

%% Replace the default \alph command with \alphalph (from the alphalph package) so that
%% \alph can handle values higher than 26.  This is useful in situations where you have
%% more than 26 subfigures in a figure.  Or more than 26 appendices.
\renewcommand{\alph}[1]{\alphalph{\value{#1}}} 
\else \fi

%% When iterating with an advisor or other reader, it may be helpful to
%% highlight content changes with change-bars.  This can be accomplished
%% using the changebar package.  When preparing the document for other
%% audiences, the change-bars can be turned off (in the document root file)
%% by controlling the changebars boolean.
\ifchangebars%
\usepackage{changebar}
\else%
%% If change-bars are off, define empty start/end commands.
\providecommand{\cbstart}{}%
\providecommand{\cbend}{}%
\fi% end ifchangebars

%% hyperref must (allegedly) be the _last_ package included
\ifelectronic
  %% If the dissertation is being typeset in an electronic-friendly way
  %% (controlled by the electronic document class option), we'll use
  %% the hyperref package to create a nicely linked pdf (with clickable
  %% citations, pdf bookmarks, etc.).
  \usepackage[verbose,pagebackref, pdftitle={\safetitletext}, pdfauthor={\authortext}, pdfsubject={Dissertation}]{hyperref}
  \ifpdf\else
  %% If compling with the LaTeX->dvi->ps->pdf pipeline, the dvips driver
  %% does not allow link text to wrap at the end of a line (even if you
  %% add \- hypenation instructions).
  %% This can cause lots of margin problems.
  %% The breakurl package addresses this problem, but only for \url links,
  %% not all the other types of links (like internal document links).
  %\usepackage{breakurl}
  %% A workaround is to use the breaklinks option of the hyperref package.
  %% That option is supposed to be for internal use, but it can be used to
  %% force wrapping; unfortunately, when viewed with a PDF viewer (rather
  %% than printed), the active link area will be misplaced.
  %% If good links are important to you, you might want to consider the
  %% pdflatex route.
  \hypersetup{breaklinks}
  \fi% end ifpdf
  %% In its capacity as a template, this document has some relative links
  %% to other files.
  %% To make relative file links work, we specify an empty prefix.
  \hypersetup{linkfileprefix={}}
\else
  %% If the electronic option is turned off, we'll still use the hyperref
  %% package, but we'll set it to draft mode.
  \usepackage[draft]{hyperref}
\fi % electronic
%% There some bugs in the backref package (which is used by the hyperref
%% package to provide links back from a reference to the citation location),
%% so we'll include a patch to work around the problems.
\usepackage{patches/backrefpatch}
%% The hyperref package provides the \autoref command that figures out what
%% type of entity you are creating a reference to, and makes the name of that
%% type of entity part of the reference (and part of the hyperlink).
%% Here we define the names that we want to be used for certain types of
%% entities.
\def\chapterautorefname{{C}hapter}
\def\sectionautorefname{{S}ection}
\def\subsectionautorefname{\sectionautorefname}
\newcommand{\subfigureautorefname}{\figureautorefname}

%% Import the file of author-defined common macros
\import{header/}{commands} 
\else \fi
\ifAppendices
%%%%%%%%%%%%%%%%%%%%%%%%%%%%%%%%%%%%%%%%%%%%%%%%%%%%%%%%%%%%%%%%%%%%%%%%%%%%%%%%
%% manySubfigsheader.tex
%% 2009.01.09 - tedmunds
%%
%% This file is the header file used to perform chapter-specific configuration
%% for one chapter in the dissertation template.
%% This file is for preamble commands that are needed by the chapter (but not
%% generally needed by the document as a whole).
%% There is another file (./manySubfigsChapterHeader) that can be used to
%% perform non-preamble configuration at the start of the chapter.
%% Note that the commands issued (and packages used) here may not _conflict_
%% with the rest of the document; they should play nicely.
%% This file is included by the main header (../header/header.tex).
%%
%%%%%%%%%%%%%%%%%%%%%%%%%%%%%%%%%%%%%%%%%%%%%%%%%%%%%%%%%%%%%%%%%%%%%%%%%%%%%%%%

%% This chapter uses the graphicx package for figures.
\usepackage{graphicx}
%% This chapter uses the subfig package to break a figure into parts (on
%% separate pages) while maintaining the sub-figure labels.
\usepackage{subfig}
%% The alphalph package allows us to have more than 26 things in an alphabetically numbered
%% list
\usepackage{alphalph}

%% Replace the default \alph command with \alphalph (from the alphalph package) so that
%% \alph can handle values higher than 26.  This is useful in situations where you have
%% more than 26 subfigures in a figure.  Or more than 26 appendices.
\renewcommand{\alph}[1]{\alphalph{\value{#1}}} 
\else \fi

%% When iterating with an advisor or other reader, it may be helpful to
%% highlight content changes with change-bars.  This can be accomplished
%% using the changebar package.  When preparing the document for other
%% audiences, the change-bars can be turned off (in the document root file)
%% by controlling the changebars boolean.
\ifchangebars%
\usepackage{changebar}
\else%
%% If change-bars are off, define empty start/end commands.
\providecommand{\cbstart}{}%
\providecommand{\cbend}{}%
\fi% end ifchangebars

%% hyperref must (allegedly) be the _last_ package included
\ifelectronic
  %% If the dissertation is being typeset in an electronic-friendly way
  %% (controlled by the electronic document class option), we'll use
  %% the hyperref package to create a nicely linked pdf (with clickable
  %% citations, pdf bookmarks, etc.).
  \usepackage[verbose,pagebackref, pdftitle={\safetitletext}, pdfauthor={\authortext}, pdfsubject={Dissertation}]{hyperref}
  \ifpdf\else
  %% If compling with the LaTeX->dvi->ps->pdf pipeline, the dvips driver
  %% does not allow link text to wrap at the end of a line (even if you
  %% add \- hypenation instructions).
  %% This can cause lots of margin problems.
  %% The breakurl package addresses this problem, but only for \url links,
  %% not all the other types of links (like internal document links).
  %\usepackage{breakurl}
  %% A workaround is to use the breaklinks option of the hyperref package.
  %% That option is supposed to be for internal use, but it can be used to
  %% force wrapping; unfortunately, when viewed with a PDF viewer (rather
  %% than printed), the active link area will be misplaced.
  %% If good links are important to you, you might want to consider the
  %% pdflatex route.
  \hypersetup{breaklinks}
  \fi% end ifpdf
  %% In its capacity as a template, this document has some relative links
  %% to other files.
  %% To make relative file links work, we specify an empty prefix.
  \hypersetup{linkfileprefix={}}
\else
  %% If the electronic option is turned off, we'll still use the hyperref
  %% package, but we'll set it to draft mode.
  \usepackage[draft]{hyperref}
\fi % electronic
%% There some bugs in the backref package (which is used by the hyperref
%% package to provide links back from a reference to the citation location),
%% so we'll include a patch to work around the problems.
\usepackage{patches/backrefpatch}
%% The hyperref package provides the \autoref command that figures out what
%% type of entity you are creating a reference to, and makes the name of that
%% type of entity part of the reference (and part of the hyperlink).
%% Here we define the names that we want to be used for certain types of
%% entities.
\def\chapterautorefname{{C}hapter}
\def\sectionautorefname{{S}ection}
\def\subsectionautorefname{\sectionautorefname}
\newcommand{\subfigureautorefname}{\figureautorefname}

%% Import the file of author-defined common macros
\import{header/}{commands} 